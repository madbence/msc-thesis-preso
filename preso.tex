\documentclass[xetex]{beamer}

\usepackage{xltxtra}
\usepackage{xcolor}
\usepackage[hungarian]{babel}

\title{Diplomaterv 1.\\Valósidejű multiplayer böngészős játék fejlesztése}
\author{Dányi Bence\\\tiny{Konzulens: Imre Gábor}}

\setsansfont{Helvetica Neue LT Pro 55 Roman}

\usetheme{Rochester}
\beamertemplatenavigationsymbolsempty

\begin{document}
  \frame{\titlepage}
  \begin{frame}
    \frametitle{A feladat}
    \begin{itemize}
      \item 2D böngészős űrhajós akciójáték
      \item Realisztikus fizikai modell
      \item \textit{Szkriptelhető vezérlés}
      \item \textit{Alapszintű mesterséges intelligencia}
      \item Tesztelés
      \item Sávszélességbarát Websocket kapcsolat
      \item Igényes grafika
      \item Üzemeltetési megoldás, éles környezetbe kihelyezés
    \end{itemize}
  \end{frame}
  \begin{frame}
    \frametitle{Architektúra}
    \includegraphics[width=\textwidth]{arch}
  \end{frame}
  \begin{frame}
    \frametitle{Fizikai modell}
    \begin{itemize}
      \item Űrhajó: kiterjedt merev test
      \item Forgatónyomaték és erő hat rá
      \item A felírt differenciálegyenlet megoldása numerikus módszerekkel (Euler módszer)
    \end{itemize}
  \end{frame}
  \begin{frame}
    \frametitle{Tesztelés}
    \begin{itemize}
      \item Unit/Integration tesztek: Mocha
      \item Funkcionális programozás: tisztán $input \to output$
      \item Perzisztens adatstruktúrák
      \item Hivatkozási átlátszóság (referential transparency)
    \end{itemize}
  \end{frame}
  \begin{frame}
    \frametitle{Websocket kapcsolat}
    \begin{itemize}
      \item Determinisztikus működés: nincs szükség a teljes állapot szinkronizálására
      \item Elegendő a nem determinisztikus eseményeket elküldeni (felhasználó inputja)
      \item Lag csökkentése: lokális predikció
    \end{itemize}
  \end{frame}
  \begin{frame}
    \frametitle{Grafika}
    \begin{itemize}
      \item WebGL (OpenGL ES 2.0 alapú API, JS bindinggal)
      \item Nincs fixed pipeline: csúcspont és fragmens árnyalók
    \end{itemize}
    % TODO: KÉP
  \end{frame}
  \begin{frame}
    \frametitle{Üzemeltetés}
    \begin{itemize}
      \item Continous Integration: Travis
      \item Deploy eszköz: Ansible
      \item \texttt{yml} leírófájl a szolgáltatáshoz
      \item SSH kapcsolaton keresztül
    \end{itemize}
  \end{frame}
  \begin{frame}
    \begin{center}
      {\Large Köszönöm a figyelmet! Kérdések? }
    \end{center}
  \end{frame}
\end{document}
